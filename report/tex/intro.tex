\section{Introduction}
Phishing, a pervasive cybercrime, has significantly escalated its attacks on the
digital landscape.As an increasing number of users rely on online platforms for
banking, government services, and other critical activities, their sensitive
information, including identification numbers, usernames, passwords, and banking
credentials, is at risk of exposure to malicious actors. The ease with which
counterfeit websites can be created, mirroring the appearance and content of
legitimate ones, makes users particularly susceptible to inadvertently entering
their personal data into fraudulent domains. To mitigate this threat and safeguard
internet users from phishing attacks, a robust phishing detector or classifier is
essential. Leveraging the power of machine learning algorithms, the phishing detector
employed in this project effectively categorizes websites as either
\emph{Phishing} or \emph{Legitimate}.

The report is structured as the following: \autoref{sec:prob-formulation} formulates
the problem by introducting the data set, specifically labels and features.
\autoref{sec:method} explains feature selection and introduces chosen
machine learning models. \autoref{sec:result} shows the final results of
proposed models and compares the performance between different approaches.
\autoref{sec:conclusion} presents conclusions based on the results shown
in \autoref{sec:result}.