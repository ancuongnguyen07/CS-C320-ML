\section{Problem Formulation}

% Subsection Dataset
\subsection{Dataset}
The Phising Dataset\footnote{https://data.mendeley.com/datasets/h3cgnj8hft/1}
contains 5000 instances of phishing webpages and 5000 of legitimate webpages which
were collected from January to May 2015 and from May 2017 to June 2017. By leveraging
the browser automation framework, Selenium WebDriver, extracted features are more
precise and robust compared to the regular-expression-based approaches~\cite{CHIEW2019153}.
All webpages in the dataset originates from the following sources:
\begin{itemize}
    \item Phishing webpage: PhishTank, OpenPhish
    \item Legitimate webpage: Alexa, Common Crawl
\end{itemize}

This dataset is WEKA-ready, which means that it is in a format compatible with the
WEKA machine learning workbench.

% Subsection Label
\subsection{Label}
In this project, I will develop a machine learning (ML) model that could detect whether a given
website is a phishing or not. Each instance in the dataset is labeled as a phishing
or a legitimate webpage by a value of 1 or 0 respectively at the \emph{label} column.
In other words, the aim of this project is to develop a supervised machine learning model
to classify a given website into two classes, \emph{Phishing} and \emph{Legitimate}, based
on selected features obtained from the dataset.

\subsection{Feature}
There are 48 features in total. Basically, most features are the number of occurences of suspicious
HTML elements or characters in URL or in web context, which are discrete values. Besides, there are
continuous features which are the percentage of external hyperlinks, resources, redirects in the
HTML webpage source code. Moreover, there are binary features indicating True/False boolean
value as 1/0. For example:
\begin{itemize}
    \item \emph{DoubleSlashInPath}: Check if ``//'' exist in path of URL.
    \item \emph{NoHttps}: Check if HTTPS exist in URL.
    \item \emph{IpAddress}: Check if IP address is used in hostname part of URL.
\end{itemize}

Additionally, there are categorical features which are encoded into three integer values: -1,0,1.

A full list of features could be investigated at~\cite{CHIEW2019153}\footnote{See a summary of features
at~\autoref{tab:all_features}}.
